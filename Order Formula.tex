\documentclass{article}
\usepackage[utf8]{inputenc}
% https://dev.guenthner.xyz/hosted/tex/unicode-helper.sty/
\usepackage{unicode-helper}
\usepackage{amsmath}
\usepackage{amssymb}
\usepackage{amsfonts}
\usepackage{microtype}
\usepackage{tikz-cd}
\usepackage[english]{babel}
\usepackage{amsthm}
\usepackage{mathtools}
\usepackage{braket}
\usepackage[backend=biber,style=numeric]{biblatex}
\usepackage[pdftex]{hyperref}
\usepackage{cleveref}

\newcommand{\titlevar}{Order Formula and its Applications}
\newcommand{\authorvar}{Günthner}
\newcommand{\datevar}{Winter 2024}
\title{\titlevar}
\author{\authorvar}
\date{\datevar}
\hypersetup{
	pdftitle=\titlevar,
	pdfauthor=\authorvar,
	pdfcreationdate=\datevar,
}
\setlength{\parindent}{0pt}

\newlength{\halflength}
\setlength{\halflength}{100pt}

\newcommand{\card}[1]{\#{#1}}
\newcommand{\inv}{^{-1}}
\newcommand{\ringunits}[1]{{#1}^{\mathclap{\rule{2.5pt}{0pt}\times}\rule{0.2em}{0pt}}}
\newcommand{\ringunitsb}[1]{\ringunits{\left({#1}{\vphantom{p^k}}\right)}}
\newcommand{\ordgroup}[1]{\ord_{\rule{0pt}{9pt}\mathclap{#1}}}
\newcommand{\ordadd}[1]{\ordgroup{ℤ/{#1}ℤ}}
\newcommand{\ordmult}[1]{\ord_{\rule{0pt}{8pt}{
			\settowidth{\halflength}{$\scriptstyle \left({ℤ/{#1}ℤ}{\vphantom{p^k}}\right)$}
			\mathrlap{\rule{-0.5 \halflength}{0pt}\ringunitsb{ℤ/{#1}ℤ}}
}}}
\newcommand{\ordker}[1]{\ordgroup{\ker(\mathrlap{#1)}}}
\newcommand{\frakB}{\mathfrak{B}}
\newcommand{\frakc}{\mathfrak{c}}
\newcommand{\bigbarn}[1]{\big({#1}\big)}
\newcommand{\Bigbarn}[1]{\Big({#1}\Big)}
\newcommand{\spospart}[1]{ {(#1)} _ {\rule{1.5pt}{0pt}\mathclap +} }
\newcommand{\pospart}[1]{{ {\bigbarn{#1}} _ {\mathclap +} }}
\newcommand{\tritem}{\item[$\blacktriangleright$]}
\newcommand{\gspan}[1]{<#1>}
\DeclareMathOperator{\ordb}{ord}
\newcommand{\ord}{\mathop{\ordb}\limits}
\DeclareMathOperator{\lcm}{lcm}
\newenvironment{pg}{

}{

\medskip

}
\newcommand{\mapdefinition}[5]{
	\begin{center}
		\begin{tabular}{llll}
			$#1:$ 	&	$#2$ & $→$ & $#3$ 	\\
					&	$#4$ & $↦$ & $#5$	\\
		\end{tabular}
	\end{center}
}

\newtheorem{definition}{Definition}
\newtheorem{lemma}{Lemma}[section]
\newtheorem{theorem}{Theorem}

\newsavebox{\proveLemmaThree}
\newsavebox{\proveLemmaFive}

\crefname{lemma}{Lemma}{Lemmata}
\crefname{equation}{equation}{equations}
\crefname{theorem}{Theorem}{Theorems}
\crefname{definition}{Definition}{Definitions}

\addbibresource{Order Formula.bib}

\begin{document}
	\maketitle
	
	\tableofcontents
	
	
	\section{Introduction}\label{sec:intro}
	
	\begin{pg}
		In this paper we will be examining the following order formula:
		\begin{equation*}
			\ordmult{p^k}(n) = \ordmult{p}(n) · p^{\max(0, k - k_p(n))}
		\end{equation*}
		for $p$ an odd prime, $k$ a natural number and $k_p(n)$ a special function.
	\end{pg}
	
	\section{Proving the Order Formula} \label{sec:proof}
	
	\begin{pg}
		This first proof only targets $p≠2$. We will see a modified proof later which will clarify the situation for $p=2$.
	\end{pg}
	
	Before we tackle the formula directly, we should start by understanding $\ringunitsb{ℤ/p^kℤ}$.
	
	\subsection{Decomposition of \texorpdfstring{$\ringunitsb{ℤ/p^kℤ}$}{the multiplicative group}}
	
	\begin{pg}
		We know from Gauss' work that the group $\ringunitsb{ℤ/p^kℤ}$ is cyclic \cite{gauss}, meaning that
		\begin{equation*}
			\ringunitsb{ℤ/p^kℤ} \cong ℤ/φ(p^k)ℤ
		\end{equation*}
		with $φ(p^k)$ the Euler-$φ$-Function which counts the number of coprime integers smaller than the number. The value is $φ(p^k)=(p-1)p^k$ and we get:
		\begin{equation*}
			\ringunitsb{ℤ/p^kℤ} \cong ℤ/(p-1)p^{k-1}ℤ
		\end{equation*}
	\end{pg}
	\begin{pg}
		Next we would like to simplify $ℤ/(p-1)p^{k-1}ℤ$ for which we can use the Chinese Remainder Theorem which tells us that for two coprime integers $a,b$ we get
		\begin{equation*}
			ℤ/abℤ \cong ℤ/aℤ \oplus ℤ/bℤ
		\end{equation*}
		Now we can write
		\begin{equation} \label{eq:1}
			\ringunitsb{ℤ/p^kℤ} \cong ℤ/(p-1)p^{k-1}ℤ \cong \bigbarn{ℤ/(p-1)ℤ} \oplus \bigbarn{ℤ/p^{k-1}ℤ}
		\end{equation}
	\end{pg}
	\begin{pg}
		This is still quite abstract so let's find subgroups of $\ringunitsb{ℤ/p^kℤ}$ that correspond to the summands.
		Define the projection onto $\ringunitsb{ℤ/pℤ}$:
		\begin{equation*}
			π_k: \ringunitsb{ℤ/p^kℤ} → \ringunitsb{ℤ/pℤ}
		\end{equation*}
		with
		\begin{equation*}
			π_k(n) = n \bmod p
		\end{equation*}
	\end{pg}
	\begin{pg}
		One of the interesting subgroups is
		\begin{equation*}
			\ker(π_k) = \set{n ϵ \ringunitsb{ℤ/p^kℤ}\ : \text{ The last digit of n is $1$}}
		\end{equation*}
		The order of this group is $\card{\ker(π_k)} = p^{k-1}$. Now since \cref{eq:1} $\ker(π_k)$ must lie entirely inside $ℤ/p^{k-1}ℤ$, because its order is coprime to that of $ℤ/(p-1)ℤ$. Now since
		\begin{equation*}
			\card{\ker(π_k)} = \card{ℤ/p^{k-1}ℤ}
		\end{equation*}
		and one of the groups is contained in the other, the groups are equal and we can rephrase \cref{eq:1}:
		\begin{equation*}
			\ringunitsb{ℤ/p^kℤ} \cong ℤ/(p-1)ℤ \oplus \ker(π_k)
		\end{equation*}
	\end{pg}
	\begin{pg}
		Substituting $\ringunitsb{ℤ/pℤ}$ for $ℤ/(p-1)ℤ$ gives us
		
		\begin{lemma} Decomposition of $\ringunitsb{ℤ/p^kℤ}$
			\label{lemma:1}
			\begin{equation*}
			\ringunitsb{ℤ/p^kℤ} \cong \ringunitsb{ℤ/pℤ} \oplus \ker(π_k)
		\end{equation*}
		\end{lemma}
	\end{pg}
	
	\subsection{Applications to the formula}
	
	What can we learn from \cref{lemma:1}?
	\begin{lemma}\label{lemma:2} Multiplicativity of $\ord$
		\begin{equation*}
			\ordmult{p^k}(n) = \ordmult{p}(n) · \ordker{π_k}(n)
		\end{equation*}
	\end{lemma}
	
	Which we will prove using the more general
	
	\begin{lrbox}{\proveLemmaThree}
		\begin{minipage}{\textwidth}
			\medskip
			\begin{lemma}\label{lemma:3} Let $A, B$ be arbitrary finite groups and $(a,b) ϵ A \oplus B$
				\begin{equation*}
					\ordgroup{A \oplus B}(a,b) = \lcm\left(\ordgroup{A}(a),\ \ordgroup{B}(b)\right)
				\end{equation*}
			\end{lemma}
			\medskip
		\end{minipage}
	\end{lrbox}
	\usebox{\proveLemmaThree}
	The proof of which will be deferred to the \nameref{sec:appendix} in \cref{sec:proveLemmaThree}.
	
	\medskip
	
	\begin{pg}
		Well we know that
		\begin{equation*}
			\ordmult{p}(n) \text{ and } \ordker{π_k}(n) \text{ coprime}
		\end{equation*}
		since the orders of their respective groups is coprime. This simplifies the $\lcm$ to a product and proves \cref{lemma:2}. ($a,b$ coprime integers implies that $\lcm(a,b) = ab$)
	\end{pg}
	\begin{pg}
		Now all that remains open is $\ordker{π_k}(n)$, for this we will need the following definition:
		\begin{definition} For $p$ prime and $n ϵ \ker(π_k)$, meaning that $n \equiv 1 \pmod p$
			\begin{equation*}
				k_p(n) \coloneq \max\set{ i ϵ ℕ : n \equiv 1 \pmod{p^i} }
			\end{equation*}
		\end{definition}
		This is almost the function from the \nameref{sec:intro}, but with a reduced domain. Now we can prove
		\begin{lemma}\label{lemma:4} For $p$ prime and $n ϵ \ker(π_k)$ %(as a reminder $\ker(π_k) \subset \ringunitsb{ℤ/p^kℤ}$)
			\begin{equation*}
				\ordker{π_k}(n) = p^{\max(0, k - k_p(n))}
			\end{equation*}
		\end{lemma}
		Let us rewrite that using $\spospart{a} \coloneq \max(0,a)$:
		\begin{equation*}
			\ordker{π_k}(n) = p^{\pospart{k-k_p(n)}}
		\end{equation*}
	\end{pg}
	
	In the proof of this formula we will be using
	
	\begin{lrbox}{\proveLemmaFive}
		\begin{minipage}{\textwidth}
			\medskip
			\begin{lemma}\label{lemma:5} $n,l ϵ ℕ$, then
				\begin{equation*}
					\ordadd{l}(n) = \frac{l}{\gcd(n, l)}
				\end{equation*}
			\end{lemma}
			\medskip
		\end{minipage}
	\end{lrbox}
	\usebox{\proveLemmaFive}
	
	This lemma will again be proven in the \nameref{sec:appendix} in \cref{sec:proveLemmaFive}.
	
	\begin{pg}
		Now this lemma only deals with the additive group $ℤ/p^kℤ$ so let us start here by proving
		\begin{equation} \label{eq:2}
			\ord(t) = p^{k-k_p'(t)} 
		\end{equation}
		with the function $k_p'$ the equivalent of $k_p$ for the additive group
		\begin{equation*}
			k_p'(t) \coloneq \max\set{ i ϵ ℕ : t \equiv 0 \pmod{p^i} }
		\end{equation*}
	\end{pg}
	\begin{pg}
		To prove \cref{eq:2} we can do a simple calculation ($ν_p$ denotes the $p$-adic valuation):
		\begin{equation*}
			\begin{split}
				\ord(t) &\stackrel{\ref{lemma:5}}{=} \frac{p^k}{\gcd(p^k, t)} = \frac{p^k}{p^{\min(k, ν_p(t))}}\\
				&= p^{k-\min(k, ν_p(t))} = p^{k+\max(-k, -ν_p(t))}  \\
				&= p^{\max(0,k-ν_p(t))} = p^\pospart{k-ν_p(t)} \\
			\end{split}
		\end{equation*}
	\end{pg}
	\begin{pg}
		But how do we translate \cref{eq:2} to the multiplicative group and \cref{lemma:4}?
		\begin{lemma}\label{lemma:6} Given the isomorphism
			\mapdefinition{ι}{ℤ/p^{k-1}ℤ}{\ker(π_k)}{l}{g^l}
			and $t ϵ \ker(π_k)$:
			\begin{equation*}
				k_p(t) = k_p'(ι\inv(t))
			\end{equation*}	
		\end{lemma}
		\begin{proof} We want to show the equality by proving that any exponent—in the set over which $k_p$ is the maximum of—is also a valid exponent for $k_p'$ (and vice versa)
			\begin{pg}
				Let $t ϵ \ker(π_k)$ and $i ϵ ℕ$ with $t \equiv 1 \pmod p^i$
			\end{pg}
			\begin{pg}
				Now $ι$ restricts to an isomorphism $ι': ℤ/p^{i-1}ℤ → \ker(π_i)$. Since neutral elements are mapped to neutral elements by isomorphisms we get
				\begin{equation*}
					ι\inv(t) \equiv 0 \pmod{p^{i-1}}
				\end{equation*}
			\end{pg}
			This proof also works with $k_p$ and $k_p'$ switched, giving us the equality.
		\end{proof}
	\end{pg}
	\begin{pg}
		\begin{proof}[Proof of \cref{lemma:4}]
			\begin{equation*}
				\ordker{π_k}(n)
					 = \ordadd{p^{k-1}}(ι\inv(n)) 
					 \stackrel{\ref{eq:2}}{=} p^\pospart{k-k_p'(ι\inv(n))} 
					 \stackrel{\ref{lemma:6}}{=} p^\pospart{k-k_p(n)}
			\end{equation*}
		\end{proof}
	\end{pg}
	\begin{pg}
		Now we have almost proved the order formula, all that is missing is the $k_p$ function for all natural numbers:
		\begin{definition}\label{def:2} Generalized $k_p$-function
			\begin{equation*}
				k_p(n) \coloneq \max\set{ i ϵ ℕ : n^{θ(n)} \equiv 1 \pmod{p^i} } = k_p(n^{θ(n)})
			\end{equation*}
			with
			\begin{equation*}
				θ(n) \coloneq \ordmult{p}(π_k(n))
			\end{equation*}
		\end{definition}
	\end{pg}
	\begin{pg}
		Now we can finish the proof of the formula by calculation:
		\begin{equation*}
			\begin{split}
				\ordmult{p^k}(n) &= θ(n) · \ordmult{p^k}({n^{θ(n)}}) = θ(n) · p^\pospart{k-k_p(n^{θ(n)})} \\ 
				&= θ(n) · p^\pospart{k-k_p(n)} = \ordmult{p}(n) · p^\pospart{k-k_p(n)}
			\end{split}
		\end{equation*}
	\end{pg}
	
	\section{Proving the Order Formula for \texorpdfstring{$p=2$}{p=2}}
	
	\begin{pg}
		This proof really is almost the same as that in \cref{sec:proof}, the only difference lies in the structure of $\ringunitsb{ℤ/p^kℤ}$. While $\ringunitsb{ℤ/p^kℤ}$ is cyclic for $p≠2$, it is semicyclic for $p=2$, it has two generators. (semicyclic for $k≥3$)
	\end{pg}
	
	\subsection{Decomposition of \texorpdfstring{$\ringunitsb{ℤ/2^kℤ}$}{the multiplicative group for p=2}}
	
	\begin{pg}
		In this subsection $k$ should be assumed to be at least $3$.
	\end{pg}
	\begin{pg}
		Let us begin by finding an element of maximal order in $\ringunitsb{ℤ/2^kℤ}$:
		\begin{lemma}\label{lemma:3.1}
			\begin{equation*}
				\ordmult{2^k}(3) = 2^{k-2}
			\end{equation*}
		\end{lemma}
		\begin{proof}[Proof by Induction] \rule{0pt}{0pt} \\
			\begin{pg}
				For $k=3$ we can do a simple computation which tells us that ${\ord(3) = 2 = 2^{k-2}}$. 
			\end{pg}
			\begin{pg}
				Now assume that $\ord(3) = 2^{k-2}$ in $\ringunitsb{ℤ/2^kℤ}$. Let us try to calculate ${3^{(2^{k-2})} \bmod 2^{k+1}}$: It is either $1$ or $2^k + 1$ as it must be $\equiv 1 \pmod{2^k}$. We can show that it is $2^k + 1$ by proving
				\begin{equation}\label{eq:3}
					ν_2(3^{(2^{k-2})}-1) = k - 1
				\end{equation}
				If instead the value was $1$ the valuation would be larger.
			\end{pg}
			\begin{pg}
				We will prove \cref{eq:3} by induction. For $k=2$ we get
				\begin{equation*}
					ν_2(3^2 - 1) = ν_2(2) = 1 = 2 - 1
				\end{equation*}	
			\end{pg}
			\begin{pg}
				Now assuming that \cref{eq:3} holds for $k$ let us compute the value for $k+1$:
				\begin{equation*}
					\begin{split}
						ν_2\left(3^{(2^{k-1})} - 1\right) 
						&= ν_2\left(  (3^{(2^{k-2})})^2  -  1^2  \right) \\
						&= ν_2\left(  (3^{(2^{k-2})} - 1)(3^{(2^{k-2})} + 1)  \right) \\
						&= ν_2\left(  3^{(2^{k-2})} - 1  ) + ν_2(  3^{(2^{k-2})} + 1  \right) \\
						&= k - 1 + 1 = k \text{ assuming the right addend is $1$}						
					\end{split}
				\end{equation*}
				Now to prove that
				\begin{equation*}
					ν_2(  3^{(2^{k-2})} + 1) = 1
				\end{equation*}
				we will simply compute
				\begin{equation*}
					\begin{split}
						3^{(2^{k-2})} + 1 &\equiv (3²)^{(2^{k-3})} + 1 \pmod 4 \\
						&\equiv 1^{(2^{k-3})} + 1 \equiv 1 + 1 \equiv 2
					\end{split}
				\end{equation*}
			\end{pg}
		\end{proof}
	\end{pg}
	\begin{pg}
		Next we will prove
		\begin{lemma} \label{lemma:3.2}
			$\set{-1, 3}$ is a generating set of $\ringunitsb{ℤ/2^kℤ}$
		\end{lemma}
		\begin{proof}
			We know from \cref{lemma:3.1} that
			\begin{equation*}
				2 · \ord(3) = 2^{k-1} = \card{\ringunitsb{ℤ/2^kℤ}}
			\end{equation*}
			We also know that
			\begin{equation*}
				\begin{split}
					3^l
					\equiv
					\begin{cases}
						1 & \text{if } l \bmod 2 = 0 	\\
						3 & \text{if } l \bmod 2 = 1	\\
					\end{cases} & \pmod 8
				\end{split}
			\end{equation*}
			meaning that $3^l \not\equiv -1 \pmod {2^k}$: $-1$ is not an element of $\gspan 3$, so the group $< -1, 3>$ must have $\ord(-1) · \ord(3) = 2^{k-1}$ elements and must be equal to the entire group.
		\end{proof}
		Now we can write
		\begin{equation}\label{eq:4}
			\ringunitsb{ℤ/2^kℤ} \cong \gspan{-1} \oplus \gspan{3}
		\end{equation}
	\end{pg}
	\begin{pg}
		However for the decomposition we would like a different form, like in \cref{lemma:1}.
		Define
		\mapdefinition{π_k}{\ringunitsb{ℤ/2^kℤ}}{\ringunitsb{ℤ/4ℤ}}{n}{n \bmod 4}
	\end{pg}
	\begin{pg}
		\begin{lemma}\label{lemma:3.3} Decomposition of $\ringunitsb{ℤ/2^kℤ}$ (for $k≥3$)
			\begin{equation*}
				\ringunitsb{ℤ/2^kℤ} \cong \ringunitsb{ℤ/4ℤ} \oplus \ker(π_k)
			\end{equation*}
		\end{lemma}
		\begin{proof}
			\begin{equation*}
				\ringunitsb{ℤ/4ℤ} \cong ℤ/2ℤ \cong \gspan{-1}
			\end{equation*}
			For $\ker(π_k)$ we will show that $\gspan{-3} \triangleleft \ker(π_k)$ and $\card{\gspan{-3}} = \card{\ker(π_k)}$:
			\begin{equation*}
				π_k(-3) 
				\equiv -1 · 3 
				\equiv -1 · -1 
				\equiv 1 \pmod 4
			\end{equation*}
			Now also all powers of $-3$ will be in the kernel.
			\begin{equation*}
				\begin{split}
					\card{\gspan{-3}} &= \ord(-3) = \lcm(\ord(-1), \ord(3)) \\
					&= 2^{k-2} = \card{\ker(π_k)}
				\end{split}
			\end{equation*}
		\end{proof}
	\end{pg}
	
	\subsection{Discussing the remaining proof}
	
	\begin{pg}
		The proof that remains is entirely equivalent to the one from \cref{sec:proof}, so only an outline of the revised version will be given.
	\end{pg}
	\begin{pg}
		\cref{lemma:3.3} together with \cref{lemma:3} gives us the following formula:
		\begin{equation*}
			\ordmult{2^k}(n) = \lcm\left(\quad \ordmult{4}(n),\ \ordker{π_k}(n)\quad \right)
		\end{equation*}
		This does not—unlike \cref{lemma:2}—simplify into a product.
	\end{pg}
	\begin{pg}
		Now all steps until \cref{def:2} will translate naturally (not in the category-theoretic sense). In the definition of $k_2$ we must replace $θ$:
		\begin{definition} Generalized $k_2$-function
			\begin{equation*}
				k_p(n) \coloneq \max\set{ i ϵ ℕ : n^{θ(n)} \equiv 1 \pmod{p^i} } = k_p(n^{θ(n)})
			\end{equation*}
			with
			\begin{equation*}
				θ(n) \coloneq \ordmult{4}(π_k(n))
			\end{equation*}
		\end{definition}
	\end{pg}
	\begin{pg}
		The last step that may be difficult to simply believe is the final calculation:
		\begin{equation*}
			\begin{split}
				\ordmult{2^k}(n) 
				&\stackrel{\ref{lemma:3.2}}{=}	\ordmult{2^k}((-1)^i · 3^j) \\
				&= 								\lcm\left(\quad \ordmult{2^k}((-1)^i),\ \ordmult{2^k}(3^j)\quad \right) \\
				&\stackrel{\text{add note}}{=} 								\lcm\left(\quad \ordmult{4}(n),\ \ordker{π_k}(3^j)\quad \right) \\
				&\stackrel{\ref{lemma:4}}{=} 								\lcm\left(\quad \ordmult{4}(n),\ 2^{\pospart{k - k_2(3^j)}} \quad \right) \\
				&\stackrel{?}{=} 				\lcm\left(\quad \ordmult{4}(n),\ 2^{\pospart{k - k_2(n)}} \quad \right) \\
			\end{split}
		\end{equation*}
		Now to verify that last question mark we have to show that $k_2((-1)^i · 3^j) = k_2(3^j)$ which we can do with
		\begin{equation*}
			k_2(3^{2j}) = k_2(3^j)
		\end{equation*}
		We show this by relating indices admissible for one side to those on the other side:
		\begin{pg}
			Let $lϵℕ$ with $3^j \equiv 1 \pmod{2^l}$, then also $3^{2j} \equiv 1$.
		\end{pg}
		\begin{pg}
			Let $lϵℕ$ with $3^{2j} \equiv 1 \pmod{2^l}$, then
			\begin{equation*}
				3^j \equiv 
				\begin{cases}
					-1 \\
					\phantom{-} 1 \\
				\end{cases}
			\end{equation*}
			as those are the only square roots of $1$, however in the proof of \cref{lemma:3.2} we showed that no $3^j \equiv -1$ and hence $3^j \equiv 1$. \hfill $\square$
		\end{pg}
	\end{pg}
	\begin{pg}
		Now we have proved the order formula for $k≥3$, but the reader may verify that the formula is also valid for $k=2$. Now we can formulate the main result of this paper:
	\end{pg}
	
	
	\begin{samepage}
		\section{The Order Formula}
		
		\begin{theorem}[The Order Formula]\label{theformula}
		For $p$ prime and $k,nϵℕ$ the following equation holds:
		\begin{equation*}
			\ordmult{p^k}(n) = 
			\begin{cases}
				\phantom{\lcm\Bigl(α_k·{}}θ_p(n) · p^\pospart{k-k_p(n)} & \text{if } p≠2 \\
				\lcm\Bigl( α_k · θ_p(n), 2^\pospart{k-k_2(n)} \Bigr) & \text{if } p=2 \\
			\end{cases}
		\end{equation*}
		with
		\begin{equation*}
			θ_p(n) = 
			\begin{cases}
				\quad \ordmult{p}(n) & \text{if } p≠2	\\
				\quad \ordmult{4}(n) & \text{if } p=2	\\
			\end{cases}
		\end{equation*}
		and
		\begin{equation*}
			k_p(n) \coloneq \max\set{ i ϵ ℕ : n^{θ_p(n)} \equiv 1 \pmod{p^i} }
		\end{equation*}
		further
		\begin{equation*}
			α_k =
			\begin{cases}
				1 & \text{if } k≠1 \\
				\frac{1}{2} & \text{if } k=1 \\
			\end{cases}
		\end{equation*}
	\end{theorem}
	\end{samepage}
	
	\section{An interesting simplification}
	
	\begin{pg}
		In this section we will try to simplify the following expression (which will be relevant for perfect numbers):
		\begin{equation*}
			ν_p(q^{ν_q(n)+1} - 1)
		\end{equation*}
		with $q≠p$ prime and $nϵℕ$.
		Let $k$ equal the expression. Then we know that $q^{ν_q(n)+1} \equiv 1 \pmod{p^k}$. We can actually write $k$ as the maximum of all $k$ with this property:
		\begin{equation*}
			ν_p(q^{ν_q(n)+1}-1) = \max\set{kϵℕ : q^{ν_q(n)+1} \equiv 1 \pmod{p^k}}
		\end{equation*}
		Why? Because the condition must hold for all smaller $k$ and also the actual $k$ (the expression) must be at least as large as any of the admissible $k$.
	\end{pg}
	\begin{pg}
		Now let us rewrite the admissibility condition (for now assume $p≠2$):
		\begin{equation*}
			\begin{split}
				q^{ν_q(n)+1} \equiv 1 \pmod{p^k} &⇔ \ord(q) \mid ν_q(n) + 1 \\
				&⇔ θ_p(n) · p^\pospart{k-k_p(n)} \mid ν_q(n) + 1 \\
				&⇔ θ_p(n) \mid ν_q(n) + 1 ∧ p^\pospart{k-k_p(n)} \mid ν_q(n) + 1 \\
			\end{split}
		\end{equation*}
		Let us continue with the right side of the $∧$:
		\begin{equation*}
			\begin{split}
				p^\pospart{k-k_p(n)} \mid ν_q(n) + 1 &⇔ \pospart{k-k_p(n)}\ ≤\ ν_p(ν_q(n) + 1) \\
				&⇔ k-k_p(n)\ ≤\ ν_p(ν_q(n)+1)
			\end{split}
		\end{equation*}
		Now if $θ_p(n) \nmid ν_q(n)+1$ we get
		\begin{equation*}
			ν_p(q^{ν_q(n)+1} - 1) = 0
		\end{equation*}
		as this is the only $k$ where \nameref{theformula} does not apply.
		Now assuming that $θ_p(n) \mid ν_q(n)+1$:
		\begin{equation*}
			ν_p(q^{ν_q(n)+1}-1) = k_p(n) + ν_p(ν_q(n) + 1)
		\end{equation*}
		as this is the last $k$ such that the inequality holds.
	\end{pg}
	\begin{pg}
		Let's rewrite the condition for $p=2$:
		\begin{equation*}
			\begin{split}
				q^{ν_q(n)+1} \equiv 1 \pmod{p^k} &⇔ \ord(q) \mid ν_q(n) + 1 \\
				&⇔ \lcm\Bigl( α_k · θ_p(n), 2^\pospart{k-k_2(n)} \Bigr) \mid ν_q(n) + 1 \\
				&⇔ α_k · θ_p(n) \mid ν_q(n) + 1 ∧ 2^\pospart{k-k_2(n)} \mid ν_q(n) + 1 \\
			\end{split}
		\end{equation*}
		Again continuing with the right side of the $∧$:
		\begin{equation*}
			\begin{split}
				2^\pospart{k-k_2(n)} \mid ν_q(n) + 1 &⇔ k-k_2(n) ≤ ν_2(ν_q(n) + 1)
			\end{split}
		\end{equation*}
	\end{pg}
	\begin{pg}
		If $θ_2(n) \nmid ν_q(n) + 1$, then
		\begin{equation*}
			ν_2(q^{ν_q(n)+1} - 1) = 1
		\end{equation*}
		as $q$ is odd. If instead $θ_2(n) \mid ν_q(n) + 1$:
		\begin{equation*}
			ν_2(q^{ν_q(n)+1} - 1) = k_2(n) + ν_2(ν_q(n) + 1)			
		\end{equation*}
	\end{pg}
	\begin{pg}
		Now for the result of this section:
		\begin{lemma}\label{lemma:5.1} Let $p≠q$ prime and $nϵℕ$
			\begin{equation*}
				ν_p(q^{ν_q(n)+1} - 1) = 
				\begin{cases}
					k_p(n) + ν_p(ν_q(n) + 1)  & \text{if } θ_p(n) \mid ν_q(n) + 1 \\
					δ_{p,2} & \text{if } θ_p(n) \nmid ν_q(n) + 1 \\
				\end{cases}
			\end{equation*}
		\end{lemma}
	\end{pg}
	
	\section{Applications to the perfect numbers}
	
	\subsection{A quick introduction}
	
	\begin{pg}
		\begin{definition}
			\begin{equation*}
				σ_t(n) = Σ_{x \mid n} x^t
			\end{equation*}
		\end{definition}
		
		\begin{lemma} For $t≠0$
			\begin{equation*}
				σ_t(n) = \prod_{p \text{ prime}} \frac{p^{(ν_p(n) + 1)t} - 1}{p^t - 1}
			\end{equation*}
		\end{lemma}
		This is a result from \cite{hardy2008}.
	\end{pg}
	
	\begin{pg}
		\begin{definition} Perfect number.
			\begin{center}
				$n$ is perfect iff $2n = σ_1(n)$
			\end{center}
		\end{definition}
	\end{pg}
	
	\subsection{The \texorpdfstring{$Ξ$}{Ξ}-Function}
	
	\begin{definition} The $Ξ$-Function:
		\begin{equation*}
			Ξ_n(q,p) \coloneq 
			\begin{cases}
				k_p(n) + ν_p(ν_q(n) + 1) - ν_p(q-1) & \text{if } θ_p(n) \mid ν_q(n) + 1 \\
				δ_{p,2} & \text{if } θ_p(n) \nmid ν_q(n) + 1 \\
			\end{cases}
		\end{equation*}
	\end{definition}
	There is an implicit $-νp(q-1)$ in the lower case, which is however zero if $θ_p(n) \nmid ν_q(n) + 1$.
	Now we get
	\begin{equation*}
		Ξ_n(q,p) = ν_p\left( \frac{p^{ν_q(n)+1}-1}{q-1} \right)
	\end{equation*}
	which relates it to $σ_1$:
	\begin{equation*}
		\begin{split}
			ν_p(σ_1(n)) &= ν_p\left(  \prod_{q \text{ prime}} \frac{q^{ν_q(n)+1}-1}{q-1} \right) \\
			&= Σ_{q \text{ prime}} ν_p\left( \frac{p^{ν_q(n)+1}-1}{q-1} \right) \\
			&= Σ_{q \text{ prime}} Ξ_n(q,p)
		\end{split}
	\end{equation*}
	
	\section{Appendix}\label{sec:appendix}
	
	\subsection{\texorpdfstring{Proof of \cref{lemma:3}}{Proof of Lemma 3}}\label{sec:proveLemmaThree}
	
	\usebox{\proveLemmaThree}
	
	\begin{proof} \rule{0pt}{0pt}\\
		Let $o_a, o_b$ denote $\ordgroup{A}(a), \ordgroup{B}(b)$ respectively and let $o_{ab}$ denote $\ordgroup{A \oplus B}(a,b)$.
		\begin{pg}
			$o_a \mid o_{ab}$, since $n^{o_{ab}} = e$ (Same for $o_b$). This gives us $\lcm(o_a, o_b) \mid o_{ab}$.
		\end{pg}
		\begin{pg}
			$o_{ab} \mid \lcm(o_a, o_b)$, since $(a,b)^{\lcm(o_a, o_b)} = (a^{\lcm(o_a, o_b)}, b^{\lcm(o_a, o_b)}) = (e,e)$
		\end{pg}
	\end{proof}
	
	\subsection{\texorpdfstring{Proof of \cref{lemma:5}}{Proof of Lemma 5}}\label{sec:proveLemmaFive}
	
	\usebox{\proveLemmaFive}
	
	\begin{proof}
		A number $x$ is equal to $\ord(n)$ exactly if it is the smallest number such that $l \mid nx$. Now $n · \ord(n) = \lcm(n, l)$. The smallest multiple of $n$ that is divisible by $l$.
		Now calculate
		\begin{equation*}
			n · \frac{l}{\gcd(n,l)} = \frac{nl}{\gcd(n,l)} = \lcm(n,l)
		\end{equation*}
		telling us that $\frac{l}{\gcd(n,l)} = \ord(n)$
	\end{proof}
	
	\printbibliography
\end{document}