\documentclass{article}

\usepackage[utf8]{inputenc}
\usepackage{unicode-helper}
\usepackage{amsmath}
\usepackage{amsfonts}
\usepackage{microtype}
\usepackage{tikz-cd}
\usepackage[english]{babel}
\usepackage{amsthm}
\usepackage{mathtools}
\usepackage{braket}
\usepackage[backend=biber,style=numeric]{biblatex}
\usepackage[pdftex]{hyperref}
\usepackage{cleveref}

\newcommand{\titlevar}{Order Formula and its Applications}
\newcommand{\authorvar}{Günthner}
\newcommand{\datevar}{Winter 2024}
\title{\titlevar}
\author{\authorvar}
\date{\datevar}
\hypersetup{
	pdftitle=\titlevar,
	pdfauthor=\authorvar,
	pdfcreationdate=\datevar,
}
\setlength{\parindent}{0pt}

\newcommand{\card}[1]{\#{#1}}
\newcommand{\inv}{^{-1}}
\newcommand{\ringunits}[1]{{#1}^{\mathclap{\rule{2.5pt}{0pt}\times}\rule{0.2em}{0pt}}}
\newcommand{\ringunitsb}[1]{\ringunits{\left(#1\right)}}
\newcommand{\ordgroup}[1]{\ord_{\rule{0pt}{8pt}\mathclap{#1}}}
\newcommand{\ordadd}[1]{\ordgroup{ℤ/{#1}ℤ}}
\newcommand{\ordmult}[1]{\ord_{\rule{0pt}{8pt}{\mathclap{\ringunitsb{ℤ/{#1}ℤ}}}}}
\newcommand{\frakB}{\mathfrak{B}}
\newcommand{\frakc}{\mathfrak{c}}
\newcommand{\bigbarn}[1]{\Big({#1}\Big)}
\newcommand{\pospart}[1]{ {\bigbarn{#1}} _ {\mathclap +} }
\DeclareMathOperator{\ordb}{ord}
\newcommand{\ord}{\mathop{\ordb}\limits}
\DeclareMathOperator{\lcm}{lcm}
\newenvironment{pg}{

}{\medskip}

\newtheorem{definition}{Definition}
\newtheorem{lemma}{Lemma}
\newsavebox{\proveLemmaThree}

\crefname{lemma}{Lemma}{Lemmata}
\crefname{equation}{equation}{equations}

\addbibresource{Order Formula.bib}

\begin{document}
	\maketitle
	
	\tableofcontents
	
	
	\section{Introduction}
	
	\begin{pg}
		In this paper we will be examining the following order formula:
		\begin{equation*}
			\ordmult{p^k}(n) = \ordmult{p}(n) · p^{\max(0, k - k_p(n))}
		\end{equation*}
		for $p$ and odd prime, $k$ a natural number and $k_p(n)$ a special function.
	\end{pg}
	
	\section{Proving the Order Formula}
	
	\begin{pg}
		This first proof only targets $p≠2$. We will see a modified proof later which will clarify the situation for $p=2$.
	\end{pg}
	\begin{pg}
		Before we tackle the formula directly, we should start by understanding $\ringunitsb{ℤ/p^kℤ}$.
	\end{pg}
	\begin{pg}
		We know from Gauss' work that the group $\ringunitsb{ℤ/p^kℤ}$ is cyclic \cite{gauss}, meaning that
		\begin{equation*}
			\ringunitsb{ℤ/p^kℤ} \cong ℤ/φ(p^k)ℤ
		\end{equation*}
		with $φ(p^k)$ the Euler-$φ$-Function which counts the number of coprime integers smaller than the number. The value is $φ(p^k)=(p-1)p^k$ and we get:
		\begin{equation*}
			\ringunitsb{ℤ/p^kℤ} \cong ℤ/(p-1)p^{k-1}ℤ
		\end{equation*}
	\end{pg}
	\begin{pg}
		Next we would like to simplify $ℤ/(p-1)p^{k-1}ℤ$ for which we can use the Chinese Remainder Theorem which tells us that for two coprime integers $a,b$ we get
		\begin{equation*}
			ℤ/abℤ \cong ℤ/aℤ \oplus ℤ/bℤ
		\end{equation*}
		Now we can write
		\begin{equation} \label{eq:1}
			\ringunitsb{ℤ/p^kℤ} \cong ℤ/(p-1)p^{k-1}ℤ \cong \bigbarn{ℤ/(p-1)ℤ} \oplus \bigbarn{ℤ/p^{k-1}ℤ}
		\end{equation}
	\end{pg}
	\begin{pg}
		This is still quite abstract so let's find subgroups of $\ringunitsb{ℤ/p^kℤ}$ that correspond to the summands.
		Define the projection onto $\ringunitsb{ℤ/pℤ}$:
		\begin{equation*}
			π: \ringunitsb{ℤ/p^kℤ} → \ringunitsb{ℤ/pℤ}
		\end{equation*}
		with
		\begin{equation*}
			π(n) = n \bmod p
		\end{equation*}
	\end{pg}
	\begin{pg}
		One of the interesting subgroups is
		\begin{equation*}
			\ker(π) = \set{n ϵ \ringunitsb{ℤ/p^kℤ}\ \mid \text{ The last digit of n is $1$}}
		\end{equation*}
		The order of this group is $\card{\ker(π)} = p^{k-1}$. Now since \cref{eq:1} $\ker(π)$ must lie entirely inside $ℤ/p^{k-1}ℤ$, because its order is coprime to that of $ℤ/(p-1)ℤ$. Now since
		\begin{equation*}
			\card{\ker(π)} = \card{ℤ/p^{k-1}ℤ}
		\end{equation*}
		the groups are equal and we can rephrase \cref{eq:1}:
		\begin{equation*}
			\ringunitsb{ℤ/p^kℤ} \cong ℤ/(p-1)ℤ \oplus \ker(π)
		\end{equation*}
	\end{pg}
	\begin{pg}
		Substituting $\ringunitsb{ℤ/pℤ}$ for $ℤ/(p-1)ℤ$ gives us
		
		\begin{lemma} Decomposition of $\ringunitsb{ℤ/p^kℤ}$
			\label{lemma:1}
			\begin{equation*}
			\ringunitsb{ℤ/p^kℤ} \cong \ringunitsb{ℤ/pℤ} \oplus \ker(π)
		\end{equation*}
		\end{lemma}
	\end{pg}
	
	\subsection{Applications to the formula}
	
	What can we learn from \cref{lemma:1}?
	\begin{lemma}\label{lemma:2} Multiplicativity of $\ord$
		\begin{equation*}
			\ordmult{p^k}(n) = \ordmult{p}(n) · \ordgroup{\ker(π)}(n)
		\end{equation*}
	\end{lemma}
	
	Which we will prove using the more general
	
	\begin{lrbox}{\proveLemmaThree}
		\begin{minipage}{\textwidth}
			\medskip
			\begin{lemma}\label{lemma:3} Let $A, B$ be arbitrary finite groups and $(a,b) ϵ A \oplus B$
				\begin{equation*}
					\ordgroup{A \oplus B}(a,b) = \lcm\left(\ordgroup{A}(a),\ \ordgroup{B}(b)\right)
				\end{equation*}
			\end{lemma}
			\medskip
		\end{minipage}
	\end{lrbox}
	\usebox{\proveLemmaThree}
	The proof of which will be deferred to the \nameref{sec:appendix} in \nameref{sec:proveLemmaThree}.
	
	\medskip
	
	\begin{pg}
		Well we know that
		\begin{equation*}
			\ordmult{p}(n) \text{ and } \ordgroup{\ker(π)}(n) \text{ coprime}
		\end{equation*}
		since the orders of their respective groups is coprime. This simplifies the $\lcm$ to a product and proves \cref{lemma:2}.
	\end{pg}
	\begin{pg}
		Now all that is open is $\ordgroup{\ker(π)}(n)$.
		We know that $\ordgroup{\ker(π)}(n) \mid p^{k-1}$
	\end{pg}
	
	\section{Appendix}\label{sec:appendix}
	
	\subsection{\texorpdfstring{Proof of \cref{lemma:3}}{Proof of Lemma 3}}\label{sec:proveLemmaThree}
	
	\usebox{\proveLemmaThree}
	
	\begin{proof} \rule{0pt}{0pt}\\
		Let $o_a, o_b$ denote $\ordgroup{A}(a), \ordgroup{B}(b)$ respectively and let $o_{ab}$ denote $\ordgroup{A \oplus B}(a,b)$.
		\begin{pg}
			$o_a \mid o_{ab}$, since $n^{o_{ab}} = e$ (Same for $o_b$). This gives us $\lcm(o_a, o_b) \mid o_{ab}$.
		\end{pg}
		\begin{pg}
			$o_{ab} \mid \lcm(o_a, o_b)$, since $(a,b)^{\lcm(o_a, o_b)} = (a^{\lcm(o_a, o_b)}, b^{\lcm(o_a, o_b)}) = (e,e)$
		\end{pg}
	\end{proof}
	
	\printbibliography
\end{document}