\documentclass{article}

\usepackage[utf8]{inputenc}
\usepackage{unicode-helper}
\usepackage{amsmath}
\usepackage{amsfonts}
\usepackage{microtype}
\usepackage{tikz-cd}
\usepackage[english]{babel}
\usepackage{amsthm}
\usepackage{mathtools}
\usepackage{braket}
\usepackage[pdftex]{hyperref}
\usepackage{cleveref}

\newcommand{\titlevar}{Order Formula and its Applications}
\newcommand{\authorvar}{Günthner}
\newcommand{\datevar}{Winter 2024}
\title{\titlevar}
\author{\authorvar}
\date{\datevar}
\hypersetup{
	pdftitle=\titlevar,
	pdfauthor=\authorvar,
	pdfcreationdate=\datevar,
}
\setlength{\parindent}{0pt}

\newcommand{\card}[1]{\#{#1}}
\newcommand{\inv}{^{-1}}
\newcommand{\ringunits}[1]{{#1}^{\mathclap{\rule{2.5pt}{0pt}\times}}}
\newcommand{\ringunitsb}[1]{\ringunits{\left(#1\right)}}
\newcommand{\ordadd}[1]{\ord_{\mathclap{ℤ/{#1}ℤ}}}
\newcommand{\ordmult}[1]{\ord_{\rule{0pt}{8pt}{\mathclap{\ringunitsb{ℤ/{#1}ℤ}}}}}
\newcommand{\frakB}{\mathfrak{B}}
\newcommand{\frakc}{\mathfrak{c}}
\newcommand{\bigbarn}[1]{\Big({#1}\Big)}
\newcommand{\pospart}[1]{ {\bigbarn{#1}} _ {\mathclap +} }
\DeclareMathOperator{\ordb}{ord}
\newcommand{\ord}{\mathop{\ordb}\limits}
\DeclareMathOperator{\lcm}{lcm}
\newenvironment{pg}{

}{\medskip}

\newtheorem{definition}{Definition}
\newtheorem{lemma}{Lemma}

\crefname{lemma}{Lemma}{Lemmata}
\crefname{equation}{equation}{equations}

\begin{document}
	\maketitle
	
	\section{Introduction}
	
	\begin{pg}
		In this paper we will be examining the following order formula:
		\begin{equation*}
			\ordmult{p^k}(n) = \ordmult{p}(n) · p^{\max(0, k - k_p(n))}
		\end{equation*}
		for $p$ and odd prime, $k$ a natural number and $k_p(n)$ a special function.
	\end{pg}
	
	\section{Notes about the proof}
	
	\begin{pg}
		
		\begin{equation*}
			Ψ_l(n) \coloneq \ordmult{l}(n)
		\end{equation*}
		
		\begin{equation*}
			n^{Ψ_{p^k}(n)} = 1 \pmod{p^k}
		\end{equation*}
		\begin{equation*}
			n^{Ψ_{p^k}(n)} = 1 \pmod{p}
		\end{equation*}
		Meaning that $Ψ_p(n) \mid Ψ_{p^k}(n)$.
	\end{pg}
	\begin{pg}
		We know that
		\begin{equation*}
			\ringunitsb{ℤ/p^kℤ} \cong \bigbarn{ℤ/(p-1)ℤ} \oplus \bigbarn{ℤ/p^{k-1}ℤ}
		\end{equation*}
		Now let
		\begin{equation*}
			π: \ringunitsb{ℤ/p^kℤ} → \ringunitsb{ℤ/pℤ}
		\end{equation*}
		\begin{equation*}
			π(x) = x \pmod p
		\end{equation*}
		Now about the kernel we know (a subgroup!):
		\begin{equation*}
			\ker π = \set{ x ϵ \ringunitsb{ℤ/p^kℤ} \ : \ x = 1 \pmod p }
		\end{equation*}
		Now since bullshit we get:
		\begin{equation*}
			\ringunitsb{ℤ/p^kℤ} / \ker π \cong \ringunitsb{ℤ/pℤ}
		\end{equation*}
		and 
		\begin{equation*}
			\ker π \cong ℤ/p^{k-1}ℤ
		\end{equation*}
	\end{pg}
	
	\section{Proving the Order Formula}
	
	\subsection{Examining Simpler Groups}
	\begin{pg}
		We would like to simply the order in one of the multiplicative groups by recursively examining smaller and smaller subgroups. For this we should first find a recursion formula for the order in groups depending on that in a smaller group.
	\end{pg}
	\begin{pg}
		Let $p$ be prime, $k ϵ ℕ$. Now let $n ϵ ℤ/p^{k+1}ℤ$ be an arbitrary element, we are interested in the following value:
		\begin{equation}\label{eq:1}
			\frac{\ordadd{p^{k+1}}(n)}{\ordadd{p^k}(n \bmod p^k)}
		\end{equation}
		the change of the order.
	\end{pg}
	
	\begin{pg}
		First we shall simplify the order in the quite simple group $ℤ/xℤ$:
		\begin{lemma}\label{lemma:1}
			\begin{equation*}
				\ordadd{x}(n) = \frac{x}{\gcd(x, n)}
			\end{equation*}
		\end{lemma}
		\begin{proof}
			We would like to show that for $t ϵ  ℕ$
			\begin{equation*}
				t · l \equiv 0 \pmod x ⇔ x \mid tl ⇔ \frac{x}{\gcd(x, l)} \mid t
			\end{equation*}
			Assume that $\frac{x}{\gcd(x, l)} \mid t$, meaning that $t = \frac{x}{\gcd(x, l)} · \square$ with $\square$ denoting an unimportant value.
			Let us examine
			\begin{equation*}
				\frac{x · l}{\gcd(x, l)} = \lcm(x, l) \equiv 0 \pmod{x}
			\end{equation*}
			Now assume instead that $x \mid tl$:
			\begin{equation*}
				\frac{x}{\gcd(x, l)} \gcd(x, l) \mid tl
			\end{equation*}
			But since already $\gcd(x, l) \mid l$ and also $\frac{x}{\gcd(x, l)}$ and $l$ are coprime, it must be that
			\begin{equation*}
				\frac{x}{\gcd(x, l)} \mid t
			\end{equation*}
		\end{proof}
	\end{pg}
	\begin{pg}
		Using \cref{lemma:1} we find the following formula:
		\begin{equation*}
			\begin{split}
				\ordadd{p^k}(n) &= \frac{p^k}{\gcd(p^k,n)} = p^{k - \min(k, ν_p(n))} \\
				&= p^{k + \max(-k, -ν_p(n))} = p^{\max(0, k-ν_p(n))}
			\end{split}
		\end{equation*}
		Now let us apply that to \cref{eq:1}:
		\begin{equation*}
			\begin{split}
				\frac{\ordadd{p^{k+1}}(n)}{\ordadd{p^k}(n \bmod p^k)} 
				&=
				\frac{p^{\max(0, k+1-ν_p(n))}}
				{p^{\max(0, k-ν_p(n \bmod p^k))}} \\
				&=
				p^{\max(0, k+1-ν_p(n)) - \max(0,k-ν_p(n \bmod p^k))} \\
				&\eqcolon p^\frakc
			\end{split}
		\end{equation*}
	\end{pg}
	\begin{pg}
		If we write $n \eqcolon m + rp^k$, $m<p^k$, $r<p$ and $\max(0, x) \eqcolon x_+$ we can rewrite the formula for $\frakc$ like so:
		\begin{equation*}
			\begin{split}
				\frakc &= {\max(0, k+1-ν_p(n)) - \max(0,k-ν_p(n \bmod p^k))} \\
				&= \pospart{k+1-ν_p(m+rp^k)} - \pospart{k-ν_p(m)}
			\end{split}
		\end{equation*}
	\end{pg}
	\begin{pg}
		For the values of $\frakc$ we find the following table:
		\begin{center}
			\begin{tabular}{c|cc}
				& $r=0$ & $r≠0$ \\
				\hline
				$m=0$ & $0$ & $1$ \\
				$m≠0$ & $1$ & $1$ \\
			\end{tabular}
		\end{center}
		\begin{proof}[Proof for $m=0$ and $r=0$]
			\begin{equation*}
				\begin{split}
					\frakc &= \pospart{k+1-ν_p(0+0)} - \pospart{k-ν_p(0)} \\
					&= \pospart{k+1-∞} - \pospart{k-∞} = 0 - 0 = 0
				\end{split}
			\end{equation*}
		\end{proof}
		\begin{proof}[Proof for $m=0$ and $r≠0$]
			\begin{equation*}
				\begin{split}
					\frakc &= \pospart{k+1-ν_p(rp^k)} - \pospart{k-ν_p(m)} \\
					&= \pospart{k+1-k} - 0 = 1
				\end{split}
			\end{equation*}
		\end{proof}
		\begin{proof}[Proof for $m≠0$]
			\begin{equation*}
				\begin{split}
					\frakc &= \pospart{k+1-ν_p(m+rp^k)} - \pospart{k-ν_p(m)} \\
					&= \pospart{k+1-ν_p(m)} - \pospart{k-ν_p(m)} \\
					&= \bigbarn{k+1-ν_p(m)} - \bigbarn{k-ν_p(m)}
					= 1
				\end{split}
			\end{equation*}
			Here we are using that
			\begin{equation*}
				ν_p(a) < ν_p(b) ⇒ ν_p(a+b) = ν_p(a)
			\end{equation*}
			and that
			\begin{equation*}
				0 < a < p^k ⇒ ν_p(a) < k
			\end{equation*}
		\end{proof}
	\end{pg}
\end{document}